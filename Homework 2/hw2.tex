\documentclass[10pt]{article}

% Packages with options
\usepackage[english]{babel}
\usepackage[mathscr]{euscript}
\usepackage[margin=1in]{geometry} 
\usepackage[utf8]{inputenc}
\usepackage[small]{titlesec}

% Primary Packages
\usepackage{adjustbox, amsbsy, amsmath, amssymb, amsthm, bm, commath, chngcntr, dsfont, econometrics, fancyhdr, gensymb, graphicx, IEEEtrantools, longtable, marginnote, mathrsfs, mathtools, mdframed, natbib, parskip, pgf, setspace, subfigure, tabularx, textcomp, tikz}

% Hyperref Setup
\usepackage[pdfauthor={Manu Navjeevan},
			bookmarks=false,%
			pdftitle={Econ 103: Homework 2},%
			pdftoolbar=false,%
			pdfmenubar=true]{hyperref} 

% Rest of the setup is in the "setup/setup_long" package
\usepackage{cleveref, setup} % cleverref should be last

%%%%%%%%%%%%%%%%%%%%%%%%%%%%%%%%%%%%%%%%%%%%%

\title{Econ 103: Homework 2} %Title
\author{Manu Navjeevan}
\date{\today}

\begin{document}
\maketitle

\section*{Single Linear Regression Theory Review}%

\begin{enumerate}
	\item Recall that we define our parameters of interest \(\beta_0\) and  \(\beta_1\) as the parameters governing the  ``line of best fit'' between \(Y\) and  \(X\):
	 \begin{equation}
		 \label{eq:argmin}
		 \beta_0,\beta_1 = \arg\min_{b_0,b_1} \E[(Y - b_0 - b_1X)^2]
	.\end{equation} 
	Once we define these parameters we define the regression error term \(\eps = Y - \beta_0 - \beta_1X\) which then generates the linear model
	 \[
	    Y = \beta_0 + \beta_1X + \eps
	.\] 
	\begin{enumerate}
		\item Using the first order conditions for \(\beta_0\) and  \(\beta_1\) (set the derivatives of the right hand side of \eqref{eq:argmin} with respect to \(b_0\) and  \(b_1\) equal to zero at) show why \(\E[\eps] = \E[\eps X]=0\).
		\item Using the definition of \(\beta_0\) and  \(\beta_1\) as line of best fit parameters, give an intuitive explanation for why  \(\E[\eps] = 0\).
	\end{enumerate}
	%\item In the case that \(X\) is binary, (\(X \in \{0,1\}\)), the parameters  \(\beta_0\) and  \(\beta_1\) from the linear model
	%\[
		%Y = \beta_0 + \beta_1X + \eps,\;\;\E[\eps] = \E[\eps X] = 0
	%.\] 
	%take on a special meaning.
	%\begin{enumerate}
		%\item Use the following equalities, true for a random variable \(X\) that takes values  \(X\in \{0,1\}\), to get an expression for \(\Cov(X,Y).\)
		%\begin{align*}
			%\E[Y] &= \E[Y|X=0]\Pr(X=0) + \E[Y|X=1]\Pr(X=1)\\
			%\E[X] &= \Pr(X=1) \\
			%\E[XY] &= \E[Y|X=1]\Pr(X=1)
		%\end{align*}
		%\item Use the following expression, true for a random variable \(X\) that takes values  \(X\in \{0,1\}\), to get a simplified expression for \(\beta_1 = \frac{\Cov(X,Y)}{\Var(X)}\):
		%\[
			%\Var(X) = \Pr(X=1)\Pr(X=0) 
	 	%.\] 
		%\item Use the expressions for \(\E[Y]\) and \(\E[X]\) above, as well as the expression for \(\beta_1\) that you derived in (b) to get a simplified expression for 
		%\[
			%\beta_0 = \E[Y] - \beta_1\E[X]
		%.\] 
		%\item Use the expressions for \(\beta_0\) and  \(\beta_1\) from above as well as the linear model
			 %\[
				 %Y = \beta_0 + \beta_1X + \eps
			 %.\] 
			 %What is the predicted value of \(Y\) when  \(X=0\)? What about when  \(X=1\)?
	%\end{enumerate}
\end{enumerate}

\section*{Hypothesis Testing and Confidence Intervals}%

In the following questions, whenver running a hypothesis test, please state the null and alternative hypotheses, show some work, and state the conclusion of the test.

\begin{enumerate}
	\item In an estimated simple regression model based on \(n = 64\), the estimated slope parameter, \(\hat\beta_1\), is  \(0.310\) and the standard error of \(\hat\beta_1\) is 0.082.
	\begin{enumerate}
		\item What is \(\hat\sigma_{\beta_1}^2\)? Recall \(\sigma_{\beta_1}\) is the terms such that, approximately for large  \(n\),
		\[
			\sqrt{n}(\hat\beta_1 - \beta_1)\sim N(0,\sigma_{\beta_1})
		.\] 
		\item Test the hypothesis that the slope is zero against the alternative that it is not at the \(1\%\) level of significance (\(\alpha = 0.01\)).	
		\item Test the hypothesis that the slope is negative against the alternative that it is positive at the \(1\%\) level of significance (\(\alpha = 0.01)\). 
		\item Test the hypothesis that the slope is positive against the alternative that it is negative at the \(5\%\) level of significance. What is the p-value?
		\item Generate a \(99\%\) confidence interval for  \(\beta_1\). How can we use this interval to run the hypothesis test in part (b)?
	\end{enumerate}
	\item Consider a simple regression of log-income (income is measured thousands of dollars), \(Y\), against years of education,  \(X\). After collecting a sample of size \(n=50\) we estimate the following regression equation. 
	 \[
		 \widehat Y = \hat\beta_0 + 0.0180 X
	.\]
	\begin{enumerate}
		\item Using the following information to solve for \(\hat\beta_0\) as well as the estimated variance \(\widehat\Var(\hat\beta_0)\), which is the square of the standard error. 
		\begin{itemize}
			\item The standard error of \(\hat\beta_0\) is  \(2.174\)
			\item The test statistic, \(t^*\), associated with the hypothesis test for
			 \[
				 H_0:\beta_0 = 0\hbox{ }\text{ vs. }\hbox{ }H_1:\beta_0 \neq 0
			,\]
			is equal to 1.257. 
		\end{itemize}
		\item Use the following information to solve for the standard error \(\hat\beta_1\) as well as the estimated variance \(\widehat\Var(\hat\beta_1)\), which is the square of the standard error.
		\begin{itemize}
			\item The test statistic, \(t^*\), associated with the hypothesis test for
			 \[
				 H_0:\beta_1 = 0\hbox{ }\text{ vs. }\hbox{ }H_1:\beta_1 \neq 0
			,\]
			is equal to \(5.754\)
		\end{itemize}
		\item Given that \(Y\) is a logged variable,  \(Y = \log(\text{income})\), how do we interpret  \(\hat\beta_1\)?
		\item Suppose that we are interested in the average value of log-income for someone with \(16\) years of education. We want to use the model above to test the hypothesis that the average value of log-income for someone with 16 years of education is less than or equal to 1.85. That is we want to test
		\[
			H_0: \lambda = \beta_0 + 16\beta_1 \leq 1.85 \hbox{ }\text{ vs. }\hbox{ }H_1: \lambda = \beta_0 + 16\beta_1 > 1.85
		.\]
		Use the fact that \(\widehat\Cov(\hat\beta_0,\hat\beta_1) = 2.84\) to test this hypothesis at level \(\alpha = 0.1\).
		\item Use the above to generate a \(90\%\) confidence interval for \(\lambda\).
	\end{enumerate}
	\item (\red{Challenge}) Suppose we find that \(\hat\beta_1 > 0\). If we reject the null hypothesis that  \(\beta_1 = 0\) in favor of an alternative hypothesis that  \(\beta_1 \neq 0\) at level \(\alpha\), up to what level can we be sure that would we reject the null hypothesis that \(\beta_1 \leq 0\) against an alternative that \(\beta_1 > 0\)? (Please give some explanation here as well as your answer, which will be some multiple of \(\alpha\)).
\end{enumerate}

\section*{\(R^2\) and Goodness of Fit}%
\begin{enumerate}
	\item Consider the following estimated regression equation.
	\[
	    \widehat Y = 6.83 + 0.869 X 
	.\]
	Write the estimated regression equation that would result if
	\begin{enumerate}
		\item All values of \(X\) were divided by 20 before estimation.
		\item All values of \(Y\) were divided by 20 before estimation. 
		\item All values of \(X\) and  \(Y\) were divided by 20 before estimation.
	\end{enumerate}
	\item Given the quantities in the questions below, calculate and interpret \(R^2\):
	 \begin{enumerate}
		 \item \( \sum_{i=1}^n (Y_i - \bar Y)^2 = 631.63\) and \( \sum_{i=1}^n \hat\eps_i^2 = 182.85\).
		 \item \( \sum_{i=1}^n Y_i^2 = 5930.94\), \(\bar Y = 16.035\),  \(n = 20\), and  \(\text{SSR} = 666.72\).
	\end{enumerate}
	\item Suppose \(R^2 = 0.7911\),  \(\text{SST}=552.36\), and \(n=20\). Find \(\hat\sigma_\eps^2\).
\end{enumerate}




\end{document}

