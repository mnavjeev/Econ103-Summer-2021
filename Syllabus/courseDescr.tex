\documentclass[10pt]{article}

% Packages with options
\usepackage[english]{babel}
\usepackage[mathscr]{euscript}
\usepackage[margin=1in]{geometry} 
\usepackage[utf8]{inputenc}
\usepackage[small]{titlesec}

% Primary Packages
\usepackage{adjustbox, amsbsy, amsmath, amssymb, amsthm, bm, commath, chngcntr, dsfont, econometrics, fancyhdr, gensymb, graphicx, IEEEtrantools, longtable, marginnote, mathrsfs, mathtools, mdframed, natbib, parskip, pgf, setspace, subfigure, tabularx, textcomp, tikz}

% Hyperref Setup
\usepackage[pdfauthor={Manu Navjeevan},
			bookmarks=false,%
			pdftitle={Econ 103 Summer Syllabus},%
			pdftoolbar=false,%
			pdfmenubar=true]{hyperref} %hyperref needs to be last

% Rest of the setup is in the "Manu" package
\usepackage{manu}

%%%%%%%%%%%%%%%%%%%%%%%%%%%%%%%%%%%%%%%%%%%%%

\title{Econ 103 Summer Syllabus}%Title
\author{Manu Navjeevan}
\date{\today}

\begin{document}
\maketitle

{\bf \Large Econ 103 Summer Session C Syllabus}\\{\small Instructor: Manu Navjeevan, Prerequisites: C- or better in Econ 41 \\ Timing: August 2nd - September 9th} 


{\bf \large Introduction:}

This course is intended to serve as an introduction to econometric modeling and analysis for upper level (sophomores and beyond) economics students. By the end of the course, students should be comfortable setting up simple econometric models, estimating the parameters of these models using data, and interpreting the estimated parameters and predictions of these models. A primary goal is to encourage students to think critically about inferences made from data that they may see in their daily lives, such as in news articles or online.

{\bf \large Course Outline:}

The course will roughly follow the outline below: 

\begin{enumerate}
  \item {\bf Week 1: Econ 41 Review}

  {\it Topics to be covered: Expectation, Variance and Covariance, Conditional Expectation, Law of Large Numbers, Central Limit Theorem }

  The goal of this section will be to introduce objects of the underlying random variables that we may be interested in estimating as well as establishing some tools (LLN and CLT) that will allow us to use data to estimate these objects of interest.

  \item {\bf Weeks 2-3: Simple Linear Regression}

  {\it Topics to be covered: Simple linear regression model, least squares estimators of model parameters, hypothesis testing, $R^2$ and forecast error}

  Weeks 2 and 3 will be spent introducing the simple linear regression model and how to use data to estimate the parameters of this model, conduct inference for the estimated parameters, interpret these estimates, and evaluate the usefulness of the model.

  \item {\bf Weeks 4-5: Multiple Linear Regression}

  {\it Topics to be covered: Multiple linear regression, regression with higher order terms, interaction terms, adjusted $R^2$, F-tests, and basics of model selection}

  Weeks 4 and 5 will be spent extending the simple linear regression model to incorporate multiple explanatory variables and higher order terms of existing explanatory variables, estimate the larger model, and conduct inference on the estimated parameters. After this basics of model selection (how to choose which variables to include) will be discussed. 

  \item {\bf Week 6: Beyond Linear Regression}

  {\it (Potential) Topics to be covered: Instrumental Variables, Differences in Differences, Potential outcomes framework, non-linear models} 

  The final week will be spent discussing how to extend the tools learned in the course to estimate more flexible models using data, issues in using observational data to estimate causal effects, and potential routes to resolve these issues.
\end{enumerate}

{\bf \large Evaluation}

Students will be graded via the following scheme:
\begin{enumerate}
  \item 60\%; problem sets (3): Bi-weekly problem sets will be graded mostly on completion.
  \begin{itemize}
  	\item Homework 1: Assigned Monday 08/09 and due Monday 08/16
	\item Homework 2: Assigned Monday 08/16 and due Monday 08/23
	\item Homework 3: Assigned Monday 08/30 and due Monday 09/06
  \end{itemize}
  \item 20\%; midterm: To be administered on Wednesday, August 25th of Week 4. 
  \item 20\%; data exercise: Administered in place of a final exam. Will give students a data set and ask them to come up with an appropriate statistical model, estimate and make inferences on the model, and interpret the estimated model. Will be assigned at the beginning of Week 6 (09/06) and due the Monday after class ends (09/13).
\end{enumerate}

\section*{Online Accommodations}

\begin{itemize}
	\renewcommand{\labelitemi}{\(\rhd\)}
	\item All lectures, labs, office hours, and TA sections will take place on zoom. You should have access to a computer with a camera, speakers, and microphone. The zoom links to the main lecture and office hours are posted on CCLE. 
	\item Lectures will be held live during their scheduled time. All lectures will be recorded and subsequently posted online to accommodate students in different time zones. Discussion sections and office hours will also be scheduled to accommodate students in different time zones
\end{itemize}

\section*{Web Page}

The official course website can be found via CCLE. Some course materials may also be found on the \href{https://github.com/mnavjeev/Econ103-Summer-2021}{GitHub} page for the class: (\verb|https://github.com/mnavjeev/Econ103-Summer-2021|). 

\section*{Center for Accessible Education}

Students needing academic accommodations based on a disability must contact the Center for Accessible Education (CAE), which will be in charge of administering assessments.

Any such arrangements with CAE must be communicated to the instructor during the first two weeks of classes. For additional information please visit the \href{https://www.cae.ucla.edu/}{CAE Website} (\verb|https://www.cae.ucla.edu|).

\section*{Academic Honesty}

We will be monitoring any code and write-ups submitted for suspicious similarities and for duplicate submissions. While we encourage students to learn from each other and form study groups, we expect that \emph{all} work and analysis submitted (problem sets and the final project) be original. Having a successful career working with data requires becoming familiar with programming languages and being able to do basic analyses on your own. For this reason it is imperative that students do the assignments on their own and do not copy and paste the work/code of their classmates.  

Any violations of academic honesty will be reported to the Office of the Dean of Students. For details about the consequences of academic integrity violations please refer \href{https://www.deanofstudents.ucla.edu/Academic-Integrity/}{here}.


\end{document}

